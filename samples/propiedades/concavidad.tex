Sea $f(x)$ una función derivable en $(a,b)$. La gráfica de $f(x)$ es convexa $(\bigcup)$ en $(a,b)$ si $f'(x)$ es creciente en $(a,b)$, y es cóncava $(\bigcap)$ si $f'(x)$ es decreciente en $(a,b)$.\\
\begin{definition}
Un \textbf{punto de inflexión} de una función es un punto en el que la función cambia de convexa $(\bigcup)$ a cóncava $(\bigcap)$, o viceversa, y la tangente atraviesa la gráfica.
\end{definition}
A continuación se muestra un ejemplo de punto de inflexión.
\geogebra{kzzn4gj2}


\begin{definition}
Estudiar la \textbf{curvatura} de una función consiste en estudiar en qué intervalos es convexa $(\bigcup)$ y en cuáles es cóncava $(\bigcap)$. Los intervalos de curvatura están separados por los puntos de inflexión y las discontinuidades.
\end{definition}

\subsubsection{Procedimiento para calcular la curvatura}
\youtube{I2T9322j-c4}
\subsubsection{Ejercicios}
\begin{ex}[sol later]
	Calcula los puntos de inflexión de las siguientes funciones:\\
	\begin{itemize}
		\item $f(x)=-x^3+3x$
		\geogebra{tkhxm6jf}
	\end{itemize}
	\begin{sol}
		\begin{itemize}
			\item Punto de inflexión en $(0,0)$.
		\end{itemize}
	\end{sol}
\end{ex}



