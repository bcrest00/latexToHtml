Sea $f(x)$ una función derivable en $(a,b)$. La gráfica de $f(x)$ es convexa $(\bigcup)$ en $(a,b)$ si $f'(x)$ es creciente en $(a,b)$, y es cóncava $(\bigcap)$ si $f'(x)$ es decreciente en $(a,b)$.\\
\begin{definition}
Un \textbf{punto de inflexión} de una función es un punto en el que la función cambia de convexa $(\bigcup)$ a cóncava $(\bigcap)$, o viceversa, y la tangente atraviesa la gráfica.
\end{definition}

\subsubsection{Procedimiento para hallar los puntos de inflexión}
\begin{enumerate}
	\item Se calcula la segunda derivada $f''(x)$.
	\item Se resuelve la ecuación, $f''(x)=0$.
	\item Se sustituyen las raíces de $f''(x)=0$ en la función inicial $y=f(x)$, y se obtienen los posibles puntos de inflexión.
	\item Se halla la tercera derivada, $f'''(x)$.
	\item Se sustituyen las abscisas de los posibles puntos de inflexión en la 3ª derivada, $f'''(x)$. Si $f'''(x) \neq 0$, son puntos de inflexión.
\end{enumerate}
\youtube{TWrj32omRRY}
\subsubsection{Criterio para el estudio de la curvatura}
Sea $f(x)$ una función cuya segunda derivada existe en $(a,b)$:
\begin{itemize}
	\item Si $f''(x) > 0$ para todo $x \in (a,b)$, la gráfica de $f(x)$ es convexa $(\bigcup)$ en $(a,b)$.
	\item Si $f''(x) < 0$ para todo $x \in (a,b)$, la gráfica de $f(x)$ es cóncava $(\bigcap)$ en $(a,b)$.
\end{itemize}

\begin{definition}
Estudiar la \textbf{curvatura} de una función consiste en estudiar en qué intervalos es convexa $(\bigcup)$ y en cuáles es cóncava $(\bigcap)$. Los intervalos de curvatura están separados por los puntos de inflexión y las discontinuidades.
\end{definition}

\subsubsection{Procedimiento para calcular la curvatura}
\begin{enumerate}
	\item Se calculan los puntos de inflexión.
	\item Se hallan las discontinuidades.
	\item Se representan en la recta real $\mathbb{R}$ las abscisas de los puntos de inflexión y las discontinuidades.
	\item Se prueba un punto de cada intervalo en la segunda derivada; solamente se considera el signo.\\
	En los intervalos consecutivos, $f''(x)$ cambia de signo si la multiplicidad de la raíz de $f''(x)$ o de su discontinuidad es impar; si es par, no cambia.
	\item Se escriben los intervalos de convexidad $(\bigcup)$, que son los correspondientes a $f''(x) > 0$.
	\item Se escriben los intervalos de concavidad $(\bigcap)$, que son los correspondientes a $f''(x) < 0$.
\end{enumerate}
\youtube{I2T9322j-c4}
\subsubsection{Ejercicios}
\begin{ex}[sol later]
	Calcula los puntos de inflexión de las siguientes funciones:\\
	\begin{itemize}
		\item $f(x)=-x^3+3x$
		\item $f(x)=x^4-2x^2-8$
		\item $f(x)=e^{-x^2}$
	\end{itemize}
	\begin{sol}
		\begin{itemize}
			\item Punto de inflexión en $(0,0)$.
			\item Puntos de inflexión en $(-\dfrac{\sqrt{3}}{3},-\dfrac{77}{9})$ y $(\dfrac{\sqrt{3}}{3},\dfrac{77}{9})$.
			\item Puntos de inflexión en $(-\dfrac{\sqrt{2}}{2},e^{-\dfrac{1}{2}})$ y $(\dfrac{\sqrt{2}}{2},e^{-\dfrac{1}{2}})$.
		\end{itemize}
	\end{sol}
\end{ex}



