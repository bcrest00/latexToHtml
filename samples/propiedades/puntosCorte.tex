
Si una función corta el \textbf{eje de abscisas}, lo hace en los puntos $(x, 0)$, es decir, los puntos donde $f(x)=0$.\\
\\
Si una función corta el \textbf{eje de ordenadas}, lo hace en el punto $(0, f(0))$, si $0$ pertenece al dominio de la función.\\
\\
Una función puede cortar el eje de abscisas varias veces, una vez o ninguna, pero no puede cortar el eje de ordenadas en más de un punto, dado que en dicho caso no sería una función.\\
\\
Para calcular los puntos de corte con el eje de abscisas, se debe de igualar la función a $0$. Siendo el valor obtenido el valor de la $x$ y el 0 el valor de la $y$. En el caso de los puntos de corte del eje de ordenadas, para calcular el valor de la $y$ correspondiente al punto que posee $x=0$, se debe de calcular el valor de la función en $0$, $y=f(0)$.\\
\youtube{FnauclNt3do}
\subsubsection{Signo de una función}
Determinar el signo de una función consiste en hallar las zonas donde la función está por encima o por debajo del eje de abscisas, es decir, los valores del dominio para los cuales $f(x) > 0$ o $f(x) < 0$ .\\
\\
Para determinar el signo de la función debemos representar en el eje de abscisas los puntos de corte de la función con dicho eje y los puntos donde la función no está definida. Después, analizamos el signo de la función en los distintos trozos.
\subsubsection{Ejercicios}

\begin{ex}[sol later]
	Determina los puntos de corte con los ejes de la función $f(x)=x^2$
	\begin{sol}
		El punto de corte con los ejes es: $(0,0)$
		\geogebra{un6xmbsf}
	\end{sol}
\end{ex}

\vspace{1cm}

\begin{ex}[sol later]
	Determina los puntos de corte con los ejes de la función $f(x)=\dfrac{3x^2}{x^2+1}$
	\begin{sol}
		El punto de corte con los ejes es: $(0,0)$
		\geogebra{fd2tfayt}
	\end{sol}
\end{ex}

\vspace{1cm}

\begin{ex}[sol later]
	Determina los puntos de corte con los ejes de la función $f(x)=x^4-2x^2$
	\begin{sol}
		Los puntos de corte con los ejes son $(0,0), (\sqrt{2}, 0)$ y $(-\sqrt{2}, 0)$
		\geogebra{e4fb746d}
	\end{sol}
\end{ex}

\vspace{1cm}

\begin{ex}[sol later]
	Determina los puntos de corte con los ejes de la función $f(x)=\sqrt{x^2-1}$
	\begin{sol}
		Los puntos de corte con los ejes son $(1, 0)$ y $(-1, 0)$
		\geogebra{mrr3hqe8}
	\end{sol}
\end{ex}

\vspace{1cm}

\begin{ex}[sol later]
	Determina los puntos de corte con los ejes de la función $f(x)=\dfrac{1}{x^2-9}$
	\begin{sol}
		El punto de corte con los ejes es: $(0,-\dfrac{1}{9})$
		\geogebra{cfme2yq3}
	\end{sol}
\end{ex}

\vspace{1cm}





\begin{ex}[sol later]
	Determina los puntos de corte con los ejes de la función que se muestra a continuación:\\
	\geogebra{mvmuqk5u}
	\begin{sol}
		El punto de corte con los ejes es: $(0,1)$
	\end{sol}
\end{ex}

\vspace{1cm}

\begin{ex}[sol later]
	Determina los puntos de corte con los ejes de la función que se muestra a continuación:\\
	\geogebra{d4audnb2}
	\begin{sol}
		No hay puntos de corte con los ejes.
	\end{sol}
\end{ex}

\vspace{1cm}

\begin{ex}[sol later]
	Determina los puntos de corte con los ejes de la función que se muestra a continuación:\\
	\geogebra{mv8baxdm}
	\begin{sol}
		No hay puntos de corte con los ejes.
	\end{sol}
\end{ex}

\vspace{1cm}

\begin{ex}[sol later]
	Determina los puntos de corte con los ejes de la función que se muestra a continuación:\\
	\geogebra{kcbmtz7k}
	\begin{sol}
		El punto de corte con los ejes es: $(0,0)$
	\end{sol}
\end{ex}

\vspace{1cm}

\begin{ex}[sol later]
	Determina los puntos de corte con los ejes de la función que se muestra a continuación:\\
	\geogebra{fw3xywww}
	\begin{sol}
		El punto de corte con los ejes es: $(0,0)$
	\end{sol}
\end{ex}
