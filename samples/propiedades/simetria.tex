
La gráfica de representación de una función puede presentar dos tipos de simetría: simetría respecto al eje de ordenadas o simetría respecto al origen de coordenadas. Veamos qué condiciones deben verificarse para que una función tenga alguno de estos tipos de simetría.
\youtube{UlD9kTKo7c8}
\subsubsection{Simetría respecto del eje de ordenadas}
\begin{definition}
	La gráfica de una función $f$ es \textbf{simétrica respecto del eje de ordenadas} si $f(x) = f(-x)$.
\end{definition}
Una función cuya gráfica es simétrica respecto del eje de ordenadas se denomina \textbf{función par}.\\
Por ejemplo, la función $f(x)=x^{2}$, es \textbf{simétrica respecto del eje de ordenadas}, puesto que:
$$f(x) = x^2 = (-x)^2 = f(-x)$$
\geogebra{cvscmh26}
\subsubsection{Simetría respecto del origen de coordenadas}
\begin{definition}
	La gráfica de una función $f$ es \textbf{simétrica respecto del origen de coordenadas} si $f(-x) = -f(x)$.
\end{definition}
Una función cuya gráfica es simétrica respecto del origen de coordenadas se denomina \textbf{función impar}.\\
Por ejemplo, la función $f(x)=x^{3}$, es \textbf{simétrica respecto del origen de coordenadas} ya que:
$$f(-x)=(-x)^3=-x^3=-f(x)$$
\geogebra{a4kksftu}
\youtube{kabnjBXPsWU}

\subsubsection{Ejercicios}
\begin{ex}[sol later]
	Estudia la simetría de las siguientes funciones indicando las operaciones que has realizado. Para ayudarte, puedes servirte de la representación gráfica de las mismas:\\
	\begin{itemize}
		\item $f(x) = 3x-x^3$
		\item $f(x) = x^4-2x^2-8$
		\item $f(x) = x^6+x^4-x^2$
		\item $f(x) = x^5+x^3-x$
		\item $f(x) = x \cdot |x|$
		\item $f(x) = |x| - 1$
		\item $f(x) = \dfrac{x^2}{1-x^2}$
		\item $f(x) = \dfrac{x}{1-x^2}$
		\item $f(x) = \dfrac{x^4+1}{x^2}$
		\item $f(x) = \dfrac{x^2}{2-x}$
	\end{itemize}
	\geogebra[ai=true, stb=true]{tc5pbmae}
	\begin{sol}
		\begin{itemize}
			\item $f(-x)=3(-x)-(-x^3)=-(3x-x^3)=-f(x) \rightarrow$ Función \textbf{simetría impar}
			\item $f(-x) = (-x)^4-2(-x)^2-8 = f(x) \rightarrow$ Función \textbf{simetría par}
			\item $f(-x) = (-x)^6+(-x)^4-(-x)^2 = x^6+x^4-x^2 = \rightarrow$ Función \textbf{simetría par}
			\item $f(-x) = (-x)^5+(-x^3)-(-x) = -x^5-x^3+x = -f(x) \rightarrow$ Función \textbf{simetría impar}
			\item $f(-x) = -x |x|= -x|x| = -f(x) \rightarrow$ Función \textbf{simetría impar}
			\item $f(-x) = |-x|-1 = |x|-1 = f(x) \rightarrow$ Función \textbf{simetría par}
			\item $f(-x)0\dfrac{(-x)^2}{1-(-x)^2} = \dfrac{x^2}{1-x^2} = f(x) \rightarrow$ Función \textbf{simetría par}
			\item $f(-x) = \dfrac{(-x)}{1-(-x)^2} = \dfrac{-x}{1-x^2} =-f(x) \rightarrow$ Función \textbf{simetría impar}
			\item $f(-x) = \dfrac{(-x)^4+1}{(-x)^2} = f(x) \rightarrow$ Función \textbf{simetría par}
			\item $f(-x) = \dfrac{(-x)^2}{2-(-x)} = \dfrac{x^2}{2+x} \rightarrow$ No presenta simetría
		\end{itemize}
	\end{sol}
\end{ex}

\vspace{1cm}


\begin{ex}[sol later]
	¿Qué tipo de simetría presentan las siguientes funciones?:\\
	\begin{itemize}
		\item $f(x) = x-1$
		\item $g(x) = \dfrac{1}{x}$
		\item $h(x) = \dfrac{x^2-4}{x^2+1}$
	\end{itemize}
	\geogebra{vyvhswrk}
	\begin{sol}
		\begin{itemize}
			\item $f(x) \rightarrow$ \textbf{no simétrica}
			\item $g(x) \rightarrow$ \textbf{simetría impar}
			\item $h(x) \rightarrow$ \textbf{simetría par}
		\end{itemize}
	\end{sol}
\end{ex}

