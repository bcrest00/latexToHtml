
\begin{definition}
Dadas dos funciones $f$ y $g$, la \emph{función compuesta} de $f$ y $g$, que se simboliza con $g \circ f$, es la función que transforma $x$ en $g(f(x))$.
$$ x \rightarrow f(x) \rightarrow g(f(x))$$
El dominio de la función compuesta $g \circ f$ está formada por los valores de $x$ pertenecientes al dominio de $f$ tales que $f(x)$ pertenece al dominio de $g$.
$$Dom(f \circ g) = Dom(f)$$
\end{definition}
Por ejemplo, si $f(x) = 3x - 1$ y $g(x) = \dfrac{1}{x^{2}+1}$, entonces la función compuesta $f$ y $g$ es $$(g \circ f)(x) = g(f(x)) = g(3x-1) = \dfrac{1}{(3x-1)^{2}+1}$$.\\
También podemos considerar la función compuesta de $g$ y $f$:
$$(f \circ g)(x) = f(g(x)) = f(\dfrac{1}{x^2 + 1}) = 3 \cdot \dfrac{1}{x^2 + 1} - 1$$
En general, la composición de funciones no es una operación conmutativa. Es decir, $g \circ f \neq f \circ g$, excepto en algunos casos particulares. Además, se puede dar el caso de que alguna de las dos funciones compuestas no exista.
\youtube{v8j1qoTvDSg}
\subsubsection{Propiedades}
Las propiedades más características de la composición de funciones son la propiedad asociativa y la propiedad no conmutativa.
\begin{itemize}
	\item \textbf{Propiedad asociativa.} Tres funciones cualesquiera $f$, $g$, $h$, que se pueden componer, verifican:
	$$h \circ (g \circ f) = (h \circ g) \circ f$$
	\item \textbf{Propiedad no conmutativa.} La composición de funciones, en general, no es conmutativa.
	$$g \circ f \neq f \circ g$$
\end{itemize}

\begin{ex}[sol later]
	Sean $f(x)=3x+2$ y $g(x)=\dfrac{x+3}{2x+1}$, calcula $(g \circ f)$ y $(f \circ g)$.
	\begin{sol}
		\begin{itemize}
			\item $g \circ f = g[f(x)] = g(3x+2) = \dfrac{3x+2+3}{2(3x+2)+1} = \dfrac{3x+5}{6x+5}$
			\item $f \circ g = f[g(x)] = f(\dfrac{x+3}{2x+1}) = 3\dfrac{x+3}{2x+1} + 2=\dfrac{7x+11}{2x+1}$
		\end{itemize}
	\end{sol}
\end{ex}

\vspace{1cm}


\begin{ex}[sol later]
	Sean $f(x)=\dfrac{x+2}{2x+1}$ y $g(x)=\sqrt{x}$, calcula $g \circ f$ y $f \circ g$.
	\begin{sol}
		\begin{itemize}
			\item $g \circ f = g[f(x)] = g(\dfrac{x+2}{2x+1}) = \sqrt{\dfrac{x+2}{2x+1}}$
			\item $f \circ g = f[g(x)] = f(\sqrt{x}) =\dfrac{\sqrt{x}+2}{2\sqrt{x}+1}$
		\end{itemize}
	\end{sol}
\end{ex}

\vspace{1cm}

\begin{ex}[sol later]
	Sean $f(x)=\dfrac{1}{2x-1}$, $g(x)=\dfrac{2x-1}{2x+1} \text{ y } h(x)=\dfrac{1}{x}$, calcula $g \circ f$, $f \circ g$, $h \circ g \circ f$ y $h \circ f \circ g$.
	\begin{sol}
		\begin{itemize}
			\item $g \circ f = g[f(x)] = g(\dfrac{1}{2x-1}) = \dfrac{2(\dfrac{1}{2x-1})-1}{2(\dfrac{1}{2x-1})+1} = \dfrac{3-2x}{2x+1}$
			\item $f \circ g = f[g(x)] = f(\dfrac{2x-1}{2x+1}) =\dfrac{1}{2(\dfrac{2x-1}{2x+1})-1} = \dfrac{2x+1}{2x-3}$
			\item $h \circ g \circ f= h[(g \circ f)(x)] = h(\dfrac{3-2x}{2x+1}) = \dfrac{2x+1}{3-2x}$
			\item $h \circ f \circ g= h[(f \circ g)(x)] = h(\dfrac{2x+1}{2x-3}) = \dfrac{2x-3}{2x+1}$
		\end{itemize}
	\end{sol}
\end{ex}

