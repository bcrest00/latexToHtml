\begin{definition}
	Una función tiene un \textbf{máximo relativo} en $x=a$ si para todo $x$ de un entorno de $a$ se verifica que $f(a)$ es mayor o igual que $f(x)$.
\end{definition}

\begin{definition}
	Una función tiene un \textbf{mínimo relativo} en $x=a$ si para todo $x$ de un entorno de $a$ se verifica que $f(a)$ es menor o igual que $f(x)$.
\end{definition}

\begin{definition}
	Una función tiene un \textbf{máximo(mínimo) absoluto} en $x=a$ si para todo $x$ ded $Dom (f)$ se verifica que $f(a)$ es mayor(menor) o igual que $f(x)$.
\end{definition}

\subsubsection{Procedimiento para calcular máximos y mínimos relativos}
\begin{enumerate}
	\item Se calcula la primera derivada, $f'(x)$.
	\item Se resuelve la ecuación $f'(x) = 0$.
	\item Se sustituyen las raíces de $f'(x) = 0$ en la función inicial $y = f(x)$ y se obtienen los posibles máximos y mínimos relativos.
	\item Se calcula la segunda derivada $f''(x)$.
	\item Se sustituyen las abscisas de los posibles máximos y mínimos relativos en la segunda derivada $f''(x)$.
\end{enumerate}

\geogebra[stb=true, ai=true]{dpeba4am}

\subsubsection{Ejercicios}
\begin{ex}[sol later]
	Clacula los máximos y los mínimos de las siguientes funciones:\\
	\begin{itemize}
		\item \geogebra{uxmbmw2k}
		\item \geogebra{mubhyvzv}
		\item \geogebra{whfwaz4t}
	\end{itemize}
	\begin{sol}
		\begin{itemize}
			\item Máximo: $(0,0)$
			\item No presenta ni máximos ni mínimos.
			\item Máximo: $(1,3)$ \\ Mínimo: $(3,-1)$
		\end{itemize}
	\end{sol}
\end{ex}

\vspace{1cm}

\begin{ex}[sol later]
	Calcula los máximos y los mínimos de las siguientes funciones:\\
	\begin{itemize}
		\item $f(x)=x^3-3x^2+3$
		\item $f(x)=3x^4-4x^3$
		\item $f(x)=\dfrac{3}{x^2+1}$
	\end{itemize}
	\begin{sol}
		\begin{itemize}
			\item Máximo en $x=0$ y mínimo en $x=2$.
			\item Mínimo en $x=1$.
			\item Máximo en $x=0$.
		\end{itemize}
	\end{sol}
\end{ex}
