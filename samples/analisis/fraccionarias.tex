
\subsubsection{Modelo de función racional}
Analiza y representa la función $y=\dfrac{x^3}{x^2-1}$
\begin{itemize}
	\item \textbf{1. Tipo de función: }racional.
	\item \textbf{2. Dominio: }por ser una función racional, hay que excluir las raíces del denominador: $x^2-1=0 \rightarrow x^2=1 \rightarrow x_1=-1$, $x_2=1$.\\
	$Dom(f) = \mathbb{R}-\{-1,1\} = (-\infty, -1) \bigcup (-1,1) \bigcup (1,+\infty)$
	\item \textbf{3. Periodicidad: }no es periódica porque las funciones racionales nunca lo son.
	\item \textbf{4. Simetrías: }$f(-x)=\dfrac{(-x)^3}{(-x)^2-1}=\dfrac{-x^3}{x^2-1}=-\dfrac{x^3}{x^2-1}$\\
	Se observa que $f(-x) = -f(x) \rightarrow$ función impar $\rightarrow$ simetría respecto al origen $O(0,0)$
	\item \textbf{5. Asíntotas: }
	\begin{itemize}
		\item Verticales: son las raíces del denominador, $x=-1$, $x=1$\\
		Posición de la curva respecto a las asíntotas verticales:\\
		$$\lim_{x \to -1^{-}}(\dfrac{x^3}{x^2-1})=\dfrac{(-1^{-})^3}{(-1^{-})^2-1}=\dfrac{-1}{0^{+}}=-\infty$$
		$$\lim_{x \to -1^{+}}(\dfrac{x^3}{x^2-1})=\dfrac{(-1^{+})^3}{(-1^{+})^2-1}=\dfrac{-1}{0^{-}}=+\infty$$
		$$\lim_{x \to 1^{-}}(\dfrac{x^3}{x^2-1})=\dfrac{(1^{-})^3}{(1^{-})^2-1}=\dfrac{1}{0^{-}}=-\infty$$
		$$\lim_{x \to 1^{+}}(\dfrac{x^3}{x^2-1})=\dfrac{(1^{+})^3}{(1^{+})^2-1}=\dfrac{1}{0^{+}}=+\infty$$
		\item Horizontales: no tiene.
		\item Oblicuas: $y=x$\\
		Posición de la curva respecto de la asíntota oblicua:
		$$\lim_{x \to -\infty}(\dfrac{x}{x^2-1})=\dfrac{-\infty}{(-\infty)^2-1}=0^{-}$$
		$$\lim_{x \to +\infty}(\dfrac{x}{x^2-1})=\dfrac{+\infty}{(+\infty)^2-1}=0^{+}$$
	\end{itemize}
	\item \textbf{6. Corte con los ejes:}
	\begin{itemize}
		\item \textbf{Eje X: }$x^3=0 \rightarrow x=0$ raíz triple. Se obtiene el punto $O(0,0)$.
		\item \textbf{Eje Y: }el punto $O(0,0)$.
		\item \textbf{Signo: }Si $x=2 \rightarrow f(2)=\dfrac{2^3}{2^2-1}=\dfrac{8}{3}>0$(+)
	\end{itemize}
	\item \textbf{7. Máximos y mínimos relativos:}\\
	\begin{itemize}
		\item $A(-\sqrt{3}, -\dfrac{3\sqrt{3}}{2})$, \textbf{máximo relativo}.\\
		\item $B(\sqrt{3}, \dfrac{3\sqrt{3}}{2})$, \textbf{mínimo relativo}.
	\end{itemize}
	\textbf{Monotonía: }\\
	\begin{itemize}
		\item Creciente: $(-\infty, -\sqrt{3}) \bigcup (\sqrt{3}, +\infty)$
		\item Decreciente: $(-\sqrt{3}, -1) \bigcup (-1, 1) \bigcup (1, \sqrt{3})$
	\end{itemize}
	\item \textbf{8. Puntos de inflexión:}\\
	\begin{itemize}
		\item $ O(0,0)$, \textbf{punto de inflexión}.
	\end{itemize}
	\textbf{Curvatura:}\\
	\begin{itemize}
		\item Convexa $(\bigcup)$: $(-1, 0) \bigcup (1, +\infty)$
		\item Cóncava $(\bigcap)$: $(-\infty, -1) \bigcup (0,1)$
	\end{itemize}	
\end{itemize}
\geogebra{s2j8hxtz}