\youtube{ARUEwJ_7sGc}
\subsubsection{Modelo de función racional}
Analiza y representa la función $y=\dfrac{x^3}{x^2-1}$
\begin{itemize}
	\item \textbf{1. Tipo de función: }racional.
	\item \textbf{2. Dominio: }por ser una función racional, hay que excluir las raíces del denominador: $x^2-1=0 \rightarrow x^2=1 \rightarrow x_1=-1$, $x_2=1$.\\
	$Dom(f) = \mathbb{R}-\{-1,1\} = (-\infty, -1) \bigcup (-1,1) \bigcup (1,+\infty)$
	\item \textbf{3. Continuidad: }es discontinua en $x=-1$, $x=1$, donde presenta discontinuidades de 1ª especie de salto finito.
	\item \textbf{4. Periodicidad: }no es periódica porque las funciones racionales nunca lo son.
	\item \textbf{5. Simetrías: }$f(-x)=\dfrac{(-x)^3}{(-x)^2-1}=\dfrac{-x^3}{x^2-1}=-\dfrac{x^3}{x^2-1}$\\
	Se observa que $f(-x) = -f(x) \rightarrow$ función impar $\rightarrow$ simetría respecto al origen $O(0,0)$
	\item \textbf{6. Asíntotas: }
	\begin{itemize}
		\item Verticales: son las raíces del denominador, $x=-1$, $x=1$\\
		Posición de la curva respecto a las asíntotas verticales:\\
		$$\lim_{x \to -1^{-}}(\dfrac{x^3}{x^2-1})=\dfrac{(-1^{-})^3}{(-1^{-})^2-1}=\dfrac{-1}{0^{+}}=-\infty$$
		$$\lim_{x \to -1^{+}}(\dfrac{x^3}{x^2-1})=\dfrac{(-1^{+})^3}{(-1^{+})^2-1}=\dfrac{-1}{0^{-}}=+\infty$$
		$$\lim_{x \to 1^{-}}(\dfrac{x^3}{x^2-1})=\dfrac{(1^{-})^3}{(1^{-})^2-1}=\dfrac{1}{0^{-}}=-\infty$$
		$$\lim_{x \to 1^{+}}(\dfrac{x^3}{x^2-1})=\dfrac{(1^{+})^3}{(1^{+})^2-1}=\dfrac{1}{0^{+}}=+\infty$$
		\item Horizontales: no tiene.
		\item Oblicuas: $y=x$\\
		Posición de la curva respecto de la asíntota oblicua:
		$$\lim_{x \to -\infty}(\dfrac{x}{x^2-1})=\dfrac{-\infty}{(-\infty)^2-1}=0^{-}$$
		$$\lim_{x \to +\infty}(\dfrac{x}{x^2-1})=\dfrac{+\infty}{(+\infty)^2-1}=0^{+}$$
	\end{itemize}
	\item \textbf{7. Corte con los ejes:}
	\begin{itemize}
		\item \textbf{Eje X: }$x^3=0 \rightarrow x=0$ raíz triple. Se obtiene el punto $O(0,0)$.
		\item \textbf{Eje Y: }el punto $O(0,0)$.
		\item \textbf{Signo: }Si $x=2 \rightarrow f(2)=\dfrac{2^3}{2^2-1}=\dfrac{8}{3}>0$(+)
	\end{itemize}
	\item \textbf{8. Máximos y mínimos relativos:}\\
	$x^2(x^2-3)=0 \rightarrow x=0 $raíz doble; $x=-\sqrt{3}$, $x=\sqrt{3}$ raíces simples.\\
	$f''(-\sqrt{3})=-\dfrac{3\sqrt{3}}{2}<0 \rightarrow A(-\sqrt{3}, -\dfrac{3\sqrt{3}}{2})$, \textbf{máximo relativo}.\\
	$f''(\sqrt{3})=\dfrac{3\sqrt{3}}{2}>0 \rightarrow B(\sqrt{3}, \dfrac{3\sqrt{3}}{2})$, \textbf{mínimo relativo}.\\
	\textbf{Monotonía: }\\
	$f'(x)=\dfrac{x^4-3x^2}{(x^2-1)^2} \rightarrow$ Si $x=2 \rightarrow f'(2)=\dfrac{2^4-3 \cdot 2^2}{(2^2-1)^2}=\dfrac{4}{9}>0$(+)\\
	Las raíces del denominador son discontinuidades dobles.
	\item \textbf{9. Puntos de inflexión:}\\
	$2x(x^2+3) = 0 \rightarrow x=0$ raíz simple.\\
	$f'''(x) = - \dfrac{6x^4+36x^2+6}{(x^2-1)^4} \rightarrow f'''(0)=-6 \neq 0 \rightarrow O(0,0)$, \textbf{punto de inflexión}.\\
	\textbf{Curvatura:}\\
	$f''(x)=\dfrac{2x^3+6x}{(x^2-1)^3} \rightarrow$ Si $x=2 \rightarrow f''(2)=\dfrac{2 \cdot 2^3+6 \cdot 6}{(2^2-1)^3}=\dfrac{28}{27}>0$(+)\\
	Las raíces del denominador son discontinuidades triples.
	
\end{itemize}
\geogebra{s2j8hxtz}