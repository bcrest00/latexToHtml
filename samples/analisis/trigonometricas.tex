\subsubsection{Modelo de función trigonométrica}
Analiza y representa la función $y=3 \sin 2x$
\begin{itemize}
	\item \textbf{1. Tipo de función: }trigonométrica.
	\item \textbf{2. Dominio: }las funciones seno y coseno están definidas en todos los números reales $\mathbb{R}$.\\
	Dom(f) = $\mathbb{R}=(-\infty, +\infty)$
	\item \textbf{3. Periodicidad: }es periódica en, de período $\dfrac{2\pi}{2} = \pi$\\
	Solo la estudiaremos en el primer período positivo $[0, \pi)$
	\item \textbf{4. Simetrías: }$f(-x) = 3 \sin (-2x) = -3 \sin (2x)$\\
	Se observa que $f(-x)=-f(x) \rightarrow$ función impar $\rightarrow$ simétrica respecto al origen de coordenadas $O(0,0)$.
	\item \textbf{5. Asíntotas: }
	\begin{itemize}
		\item Verticales: no tiene.
		\item Horizontales: no tiene.
		\item Oblicuas: no tiene.
	\end{itemize}
	\item \textbf{6. Corte con los ejes: }
	\begin{itemize}
		\item \textbf{Eje X: }$3\sin(2x)=0 \rightarrow \sin(2x) = 0 \rightarrow x_1=0$, $x_2=\dfrac{\pi}{2}$, raíces simples; Se obtienen los puntos $O(0,0); A(\dfrac{\pi}{2}, 0)$.\\
		\item \textbf{Eje Y: }es el punto $O(0,0)$
		\item \textbf{Signo: }Si $x=\dfrac{\pi}{4} \rightarrow f(\dfrac{\pi}{4})=3\cdot \sin(\dfrac{2\pi}{4}) = 3 \sin(\dfrac{\pi}{2})>0$ (+)
	\end{itemize}
	\item \textbf{7. Máximos y mínimos: }\\
	\begin{itemize}
		\item $B(\dfrac{\pi}{4}, 3)$, \textbf{máximo relativo}
		\item $C(\dfrac{3\pi}{4}, -3)$, \textbf{mínimo relativo}.
	\end{itemize}
	\textbf{Monotonía: }\\
	\begin{itemize}
		\item Creciente: $(0, \dfrac{\pi}{4}) \bigcup (\dfrac{3\pi}{4}, \pi))$
		\item Decreciente: $(\dfrac{\pi}{4}, \dfrac{3\pi}{4})$
	\end{itemize}\item 
	\item \textbf{8. Puntos de inflexión:}\\
	\begin{itemize}
		\item $O(0,0)$, \textbf{punto de inflexión}.
		\item $A(\dfrac{\pi}{2},0)$, \textbf{punto de inflexión}.
	\end{itemize}
	\textbf{Curvatura: }\\
	\begin{itemize}
		\item Convexa $(\bigcup)$: $(\dfrac{\pi}{2}, \pi)$.
		\item Cóncava $(\bigcap)$: $(0, \dfrac{\pi}{2})$
	\end{itemize}
\end{itemize}
\geogebra{jufgczka}