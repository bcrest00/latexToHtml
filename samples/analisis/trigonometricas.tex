\youtube{9YmBUGs7NMQ}
\subsubsection{Modelo de función trigonométrica}
Analiza y representa la función $y=3 \sin 2x$
\begin{itemize}
	\item \textbf{1. Tipo de función: }trigonométrica.
	\item \textbf{2. Dominio: }las funciones seno y coseno están definidas en todos los números reales $\mathbb{R}$.\\
	Dom(f) = $\mathbb{R}=(-\infty, +\infty)$
	\item \textbf{3. Continuidad: }el seno y coseno son continuas en toda la recta real $\mathbb{R}$.
	\item \textbf{4. Periodicidad: }es periódica en, de período $\dfrac{2\pi}{2} = \pi$\\
	Solo la estudiaremos en el primer período positivo $[0, \pi)$
	\item \textbf{5. Simetrías: }$f(-x) = 3 \sin (-2x) = -3 \sin (2x)$\\
	Se observa que $f(-x)=-f(x) \rightarrow$ función impar $\rightarrow$ simétrica respecto al origen de coordenadas $O(0,0)$.
	\item \textbf{6. Asíntotas: }
	\begin{itemize}
		\item Verticales: no tiene.
		\item Horizontales: no tiene.
		\item Oblicuas: no tiene.
	\end{itemize}
	\item \textbf{7. Corte con los ejes: }
	\begin{itemize}
		\item \textbf{Eje X: }$3\sin(2x)=0 \rightarrow \sin(2x) = 0 \rightarrow x_1=0$, $x_2=\dfrac{\pi}{2}$, raíces simples; Se obtienen los puntos $O(0,0); A(\dfrac{\pi}{2}, 0)$.\\
		\item \textbf{Eje Y: }es el punto $O(0,0)$
		\item \textbf{Signo: }Si $x=\dfrac{\pi}{4} \rightarrow f(\dfrac{\pi}{4})=3\cdot \sin(\dfrac{2\pi}{4}) = 3 \sin(\dfrac{\pi}{2})>0$ (+)
	\end{itemize}
	\item \textbf{8. Máximos y mínimos: }\\
	$f(x)=3 \sin(2x)$
	\begin{itemize}
		\item $f(\dfrac{\pi}{4})=3 \sin(\dfrac{\pi}{2}) = 3 \rightarrow B(\dfrac{\pi}{4}, 3)$
		\item $f(\dfrac{3\pi}{4})=3 \sin(\dfrac{3\pi}{2}) = -3 \rightarrow C(\dfrac{3\pi}{4}, -3)$
	\end{itemize}
	$f''(x)=-12 \sin(2x)$
	\begin{itemize}
		\item $f''(\dfrac{\pi}{4})=-12 <0 \rightarrow B(\dfrac{\pi}{4}, 3)$, \textbf{máximo relativo}.
		\item $f''(\dfrac{3\pi}{4})=12 >0 \rightarrow C(\dfrac{3\pi}{4}, -3)$, \textbf{mínimo relativo}.
	\end{itemize}
	\textbf{Monotonía: }\\
	$f'(x) = 6 \cos(2x) \rightarrow$ Si $x=\dfrac{\pi}{2} \rightarrow f'(\dfrac{\pi}{2}) = 6 \cos(\pi) = -6 < 0$ (-)
	\item \textbf{9. Puntos de inflexión:}\\
	$f''(x)=-12\sin(2x) \rightarrow \sin(2x) = 0$
	\begin{itemize}
		\item $2x=0 \rightarrow x=0$, raíz simple. 
		\item $2x=\pi \rightarrow x=\dfrac{\pi}{2}$, raíz simple. 
	\end{itemize}
	$f(x)=3\sin(2x)$
	\begin{itemize}
		\item $f(0)=3\sin(0)=0 \rightarrow O(0,0)$ 
		\item $f(\dfrac{\pi}{2})=3\sin(\pi)=0 \rightarrow A(\dfrac{\pi}{2},0)$  
	\end{itemize}
	$f'''(x)=-24\cos(2x)$
	\begin{itemize}
		\item $f'''(0)=-24\cos(0)=-24 \neq 0 \rightarrow O(0,0)$, \textbf{punto de inflexión}. 
		\item $f'''(\dfrac{\pi}{2})=-24\cos(\pi)=24 \neq 0 \rightarrow A(\dfrac{\pi}{2},0)$, \textbf{punto de inflexión}.
	\end{itemize}	
	\textbf{Curvatura: }\\
	$f''(x)=-12 \sin(2x) \rightarrow$ Si $x=\dfrac{\pi}{4} \rightarrow f''(\dfrac{\pi}{4})=-12 \sin (\dfrac{\pi}{2}) = -12 <0$ (-)
\end{itemize}
\geogebra{jufgczka}