Hacer el estudio sobre una gráfica de una función consiste en analizar sus características a partir de lo que se observa en la gráfica. Para realizar este estudio, es necesario llevar a cabo, ordenadamente, los pasos que se indican en la siguiente tabla.\\
\begin{table}[]
	
	\begin{tabular}{|p{5cm}|}
		\hline
	    \textbf{Formulario: características} \\ \hline
		\textbf{1. Tipo de función:} consiste en clasificar la función.\newline \\ \hline
		\textbf{2. Dominio de una función:} conjunto de valores que toma la variable inependiente \textbf{x}. Se representa por \textbf{Dom(f)}.\\ \hline
		\textbf{3. Periodicidad de una función:} una función es periódica si se repite en intervalos iguales.\\ \hline
		\textbf{4. Simetría de una función:} se estudiarán sólo las simetrías respecto del origen O(0,0) y respecto del eje Y.\\ \hline
		\textbf{5. Asíntotas de una función:} rectas a las que se acerca la función en puntos muy alejados del origen sin llegar a tocarlas. Las asíntotas pueden ser verticales, horizontales y oblicuas. \\ \hline
		\textbf{6. Puntos de corte de una función con los ejes:} puntos en que $x=0$ y/o $y=0$. La gráfica puede cortar al eje X en varios puntos; al eje Y, como máximo, en uno. \textbf{Signo:} intervalos del eje X en los que la función es positiva (+) o negativa (-). Las regiones están separadas por las abscisas de los puntos de corte del eje X y por las discontinuidades. \\ \hline
		\textbf{7. Máximos y mínimos relativos de una función:}\newline \textbf{Máximo relativo:} punto en el que el valor de la función es mayor que en los puntos que están muy cercanos.\newline \textbf{Mínimo relativo: } punto en que el valor de la función es menor que en los puntos que están muy cercanos.\newline \textbf{Monotonía: } consiste en estudiar en qué intervalos la función es creciente y en cuáles es decreciente. Los intervalos de crecimiento están separados por las abscisas de los máximos y mínimos relativos y por las discontinuidades. \\ \hline
		\textbf{8. Punto de inflexión de una función:} punto en que la función cambia de convexa $(\bigcup)$ a cóncava $(\bigcap)$ o viceversa, es decir, la recta tangente atraviesa a la gráfica.\newline \textbf{Curvatura:} consiste en estudiar en qué intervalos es convexa $(\bigcup)$ y en cuáles es cóncava $(\bigcap)$. Los intervalos de curvatura están separados por las abscisas de los puntos de inflexión y las discontinuidades.\\ \hline
		\textbf{9. Recorrido o imagen de una función:} conjunto de valores que toma la variable dependiente \textbf{y}. Se representa por \textbf{Im(f)}.\\ \hline
	\end{tabular}
	
\end{table}


\subsubsection{Ejercicios}

\begin{ex}[sol later]
	\begin{itemize}
		\item ¿Cuántos tipos de simetría de funciones existen?¿Cuáles son?
		\item ¿Cómo se denomina el punto en el que el valor de la función es menor que en los puntos cercanos a él?
		\item ¿Cómo se denomina al conjunto de valores que toma la variable independiente $x$?
		\item ¿Cómo se denomina el conjunto de valores que toma la variable dependiente $y$?
	\end{itemize}
	\begin{sol}
		\begin{itemize}
			\item Dos. Simetría respecto del origen y simetría respecto del eje $Y$.
			\item Mínimo.
			\item Dominio.
			\item Imagen.
		\end{itemize}
	\end{sol}
\end{ex}
