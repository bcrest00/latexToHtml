\youtube{akSVuDk_lhw}
\subsubsection{Modelo de función irracional}
Analiza y representa la función $y=\sqrt{x^2-4}$
\begin{itemize}
	\item \textbf{1. Tipo de función: }irracional.
	\item \textbf{2. Dominio: }por ser una función irracional de índice par, el radicando tiene que ser mayor o igual que cero.\\
	$x^2-4 \geq 0$, se resuelve la ecuación correspondiente $x^2-4=0 \rightarrow x^2=4 \rightarrow x_1=-2$, $x_2=2$. Como las raíces son simples, $x^2-4$ cambia de signo en cada una de ellas.\\
	Dom(f)= $(-\infty, -2] \bigcup [2, +\infty)$
	\item \textbf{3. Continuidad: }es discontinua en $x=-2$, $x=2$.
	\begin{itemize}
		\item Para $x=-2$, se tiene $f(-2)=0$:
		$$\lim_{x \to -2^{-}}(\sqrt{x^2-4})=0$$
		$$\lim_{x \to -2^{+}}(\sqrt{x^2-4}) \text{no existe}$$
		Por tanto, para $x=-2$, la función tiene una discontinuidad de 2ª especie.\\
		\item Para $x=-2$, se tiene $f(-2)=0$:
		$$\lim_{x \to 2^{-}}(\sqrt{x^2-4}) \text{no existe}$$
		$$\lim_{x \to 2^{+}}(\sqrt{x^2-4})=0$$
		Por tanto, para $x=2$, la función tiene una discontinuidad de 2ª especie.\\
	\end{itemize}
	\item \textbf{4. Periodicidad: }no es periódica. Las funciones irracionales nunca lo son.
	\item \textbf{5. Simetrías: }$f(-x) = \sqrt{(-x)^2-4} = \sqrt{x^2-4}$\\
	Se observa que $f(-x)=f(x) \rightarrow $función par $\rightarrow$ simétrica respecto al eje Y.
	\item \textbf{6. Asíntotas: }
	\begin{itemize}
		\item Verticales: no tiene.
		\item Horizontales: no tiene.
		\item Oblicuas: presenta asíntotas oblicuas en $y=x$ y en $y=-x$.
	\end{itemize}
	\item \textbf{7. Corte con los ejes: }
	\begin{itemize}
		\item \textbf{Eje X: }$\sqrt{x^2-4} = 0 \rightarrow x^2-4=0 \rightarrow x^2=4 \rightarrow x_1=-2$, $x_2=2$\\
		Se obtienen los puntos $A(-2, 0)$ y $B(2,0)$
		\item \textbf{Eje Y: }no lo corta.
		\item \textbf{Signo: }Si $x=3 \rightarrow f(3)=\sqrt{3^2-4}=\sqrt{9-4}=\sqrt{5}>0$ (+)
	\end{itemize}
	\item \textbf{8. Máximos y mínimos relativos: }
	$f'(x)=\dfrac{x}{\sqrt{x^2-4}} \rightarrow x=0 \notin$ Dom(f)\\
	No tiene ni máximos ni mínimos relativos.\\
	\textbf{Monotonía: }si $x=3 \rightarrow f'(3)=\dfrac{3}{\sqrt{3^2-4}}=\dfrac{3}{\sqrt{5}}>0$ (+)
	\item \textbf{9. Puntos de inflexión: }\\
	$f''(x)=-\dfrac{4}{(x^2-4)\sqrt{x^2-4}}$ \\
	$f''(x)$ nunca se hace cero, por lo tanto no ha puntos de inflexión.\\
	\textbf{Curvatura: }si $x=3 \rightarrow f''(3)=-\dfrac{4}{(3^2-4)\sqrt{3^2-4}}=-\dfrac{4}{5\sqrt{5}}<0$ (-)	
\end{itemize}
\geogebra{nxe84yaw}