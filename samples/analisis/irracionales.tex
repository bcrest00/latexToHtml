
\subsubsection{Modelo de función irracional}
Analiza y representa la función $y=\sqrt{x^2-4}$
\begin{itemize}
	\item \textbf{1. Tipo de función: }irracional.
	\item \textbf{2. Dominio: }por ser una función irracional de índice par, el radicando tiene que ser mayor o igual que cero.\\
	$x^2-4 \geq 0$, se resuelve la ecuación correspondiente $x^2-4=0 \rightarrow x^2=4 \rightarrow x_1=-2$, $x_2=2$. Como las raíces son simples, $x^2-4$ cambia de signo en cada una de ellas.\\
	Dom(f)= $(-\infty, -2] \bigcup [2, +\infty)$
	\item \textbf{3. Periodicidad: }no es periódica. Las funciones irracionales nunca lo son.
	\item \textbf{4. Simetrías: }$f(-x) = \sqrt{(-x)^2-4} = \sqrt{x^2-4}$\\
	Se observa que $f(-x)=f(x) \rightarrow $función par $\rightarrow$ simétrica respecto al eje Y.
	\item \textbf{5. Asíntotas: }
	\begin{itemize}
		\item Verticales: no tiene.
		\item Horizontales: no tiene.
		\item Oblicuas: presenta asíntotas oblicuas en $y=x$ y en $y=-x$.
	\end{itemize}
	\item \textbf{6. Corte con los ejes: }
	\begin{itemize}
		\item \textbf{Eje X: }$\sqrt{x^2-4} = 0 \rightarrow x^2-4=0 \rightarrow x^2=4 \rightarrow x_1=-2$, $x_2=2$\\
		Se obtienen los puntos $A(-2, 0)$ y $B(2,0)$
		\item \textbf{Eje Y: }no lo corta.
		\item \textbf{Signo: }Si $x=3 \rightarrow f(3)=\sqrt{3^2-4}=\sqrt{9-4}=\sqrt{5}>0$ (+)
	\end{itemize}
	\item \textbf{7. Máximos y mínimos relativos: }
	\begin{itemize}
		\item No hay máximos.
		\item No hay mínimos.
	\end{itemize}
	\textbf{Monotonía: }
	\begin{itemize}
		\item Creciente: $(2, +\infty)$
		\item Decreciente: $(-\infty, -2)$
	\end{itemize}
	\item \textbf{8. Puntos de inflexión: }\\
	\begin{itemize}
		\item No tiene puntos de inflexión.
	\end{itemize}
	\textbf{Curvatura: }
	\begin{itemize}
		\item Convexa $(\bigcup)$: nada.
		\item Cóncava $(\bigcap)$: $(-\infty, -2) \bigcup (2, +\infty)$
	\end{itemize}
\end{itemize}
\geogebra{nxe84yaw}
\subsubsection{Ejercicios}
\begin{ex}[sol later]
	Ayudándote de la gráfica, realiza el estudio de la siguiente función $f(x)=2 \cdot \sqrt{\dfrac{2}{x}-1}$.
	\geogebra{cukvrpna}
	\begin{sol}
		\begin{itemize}
			\item Tipo de función: \textbf{función irracional}.
			\item Dominio: $Dom(f) = (0,2]$.
			\item Periodicidad: no es periódica.
			\item Simetría: \textbf{no presenta simetría}.
			\item Asíntotas: verticales $(x=0)$.
			\item Puntos de corte: $\mathbf{(2,0)}$.
			\item Máximos y mínimos: mínimo en $(2,0)$.
			\item Monotonía: decreciente $(0,2]$.
			\item Punto inflexión: $(\dfrac{3}{2},\dfrac{2\sqrt{3}}{3})$.
			\item Curvatura:
			\begin{itemize}
				\item Convexa: $(0, \frac{3}{2})$
			\item Cóncava: $(\dfrac{3}{2},2)$
		\end{itemize}
	\end{itemize}
\end{sol}
\end{ex}
