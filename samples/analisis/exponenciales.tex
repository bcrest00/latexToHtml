\youtube{JulYyOS0hH4}
\subsubsection{Modelo de función exponencial}
Analiza y representa la función $y=(2-x)e^x$
\begin{itemize}
	\item \textbf{1. Tipo de función: }producto de polinómica por exponencial.
	\item \textbf{2. Dominio: }por ser el producto de una función polinómica por una función exponencial, es toda la recta real $\mathbb{R}$.\\
	Dom(f) = $\mathbb{R}=(-\infty, +\infty)$
	\item \textbf{3. Continuidad: }por ser el producto de una función polinómica por una exponencial, es continua en toda la recta real $\mathbb{R}$. 
	\item \textbf{4. Periodicidad: }no es periódica, ya que las funciones polinómicas y exponenciales nunca lo son.
	\item \textbf{5. Simetrías: }$f(-x)=(2+x)e^{-x}$\\
	Se observa que $f(-x)\neq f(x)$, $f(-x) \neq -f(x) \rightarrow$ no es simétrica ni respecto al eje Y ni respecto del orien $O(0,0)$.
	\item \textbf{6. Asíntotas: }\\
	\begin{itemize}
		\item Verticales: no tiene.
		\item Horizontales: \\
		$$\lim_{x \to -\infty}((2-x)e^x)=0$$
		$$\lim_{x \to +\infty}((2-x)e^x)=-\infty$$
		Asíntota horizontal $y=0$, pero sólo por la izquierda.\\
		Posición de la curva respecto de la asíntota oblicua:
		$$\lim_{x \to -\infty}((2-x)e^x)=0^{+}$$
		La curva está encima de la asíntota.
		\item Oblicuas: no tiene.
	\end{itemize}
	\item \textbf{7. Corte con los ejes: }
	\begin{itemize}
		\item \textbf{Eje X: }$(2-x)e^{x} = 0 \rightarrow x=2$, raíz simple. Se obtiene el punto $A(2,0)$.
		\item \textbf{Eje Y: }es el punto $B(0,2)$.
		\item \textbf{Signo: }Si $x=0 \rightarrow f(0)=2 >0$ (+)
	\end{itemize}
	\item \textbf{8. Máximos y mínimos relativos: }\\
	$f'(x)=(1-x)e^x \rightarrow (1-x)e^x = 0 \rightarrow x=1$, raíz simple.\\
	$f(x)=(2-x)e^x \rightarrow f(1)=e \rightarrow C(1,e)$\\
	$f''(x)=-ze^{x} \rightarrow f''(x)=-e < 0 \rightarrow C(1,e)$, \textbf{máximo relativo}.\\
	\textbf{Monotonía: }\\
	$f''(x)=(1-x)e^x \rightarrow $Si $x=0 \rightarrow f'(0) = 1>0$ (+) 
	\item \textbf{9. Puntos de inflexión:}\\
	$f''(x)=-xe^x \rightarrow -xe^x = 0 \rightarrow x=0$, raíz simple.\\
	$f(x)=(2-x)e^x \rightarrow f(0)=2 \rightarrow B(0,2)$\\
	$f'''(x)=-(x+1)e^x \rightarrow f'''(0)=-1 \neq 0 \rightarrow B(0,2)$, \textbf{punto de inflexión}.\\
	\textbf{Curvatura:}\\
	$f''(x)=-xe^x \rightarrow $si $x=1 \rightarrow f''(1)=-e<0$ (-)
	
\end{itemize}
\geogebra{cthuzvhc}