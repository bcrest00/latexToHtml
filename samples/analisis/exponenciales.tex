La función exponencial es una función cuya ecuación y gráfica pertenecen a uno de los siguientes tipos:
\begin{itemize}
	\item $f(x) = a^{x}; a>1$
	\geogebra{adsky8qe}
	\item $f(x) = a^{x}; 0<a<1$
	\geogebra{xuyck94f}
\end{itemize}
De la observación y estudio de las gráficas de estas familias de funciones se deducen sus propiedades:
\begin{itemize}
	\item \textbf{Dominio. }El dominio de estas funciones es el conjunto de todos los números reales:$$Dom(f) = \mathbb{R}$$
	\item \textbf{Monotonía. }Estas funciones son:
	\begin{itemize}
		\item estrictamente crecientes si $a>1$.
		\item estrictamente decrecientes si $0<a<1$.
	\end{itemize}
	\item \textbf{Extremos relativos.} Carecen de extremos relativos.
	\item \textbf{Simetría. }No presentan simetría de ningún tipo.
	\item \textbf{Asíntotas. }Tienen asíntota horizontal.
\end{itemize}
\subsubsection{Modelo de función exponencial}
Analiza y representa la función $y=(2-x)e^x$
\begin{itemize}
	\item \textbf{1. Tipo de función: }producto de polinómica por exponencial.
	\item \textbf{2. Dominio: }por ser el producto de una función polinómica por una función exponencial, es toda la recta real $\mathbb{R}$.\\
	Dom(f) = $\mathbb{R}=(-\infty, +\infty)$
	\item \textbf{3. Periodicidad: }no es periódica, ya que las funciones polinómicas y exponenciales nunca lo son.
	\item \textbf{4. Simetrías: }$f(-x)=(2+x)e^{-x}$\\
	Se observa que $f(-x)\neq f(x)$, $f(-x) \neq -f(x) \rightarrow$ no es simétrica ni respecto al eje Y ni respecto del orien $O(0,0)$.
	\item \textbf{5. Asíntotas: }\\
	\begin{itemize}
		\item Verticales: no tiene.
		\item Horizontales: \\
		$$\lim_{x \to -\infty}((2-x)e^x)=0$$
		$$\lim_{x \to +\infty}((2-x)e^x)=-\infty$$
		Asíntota horizontal $y=0$, pero sólo por la izquierda.\\
		Posición de la curva respecto de la asíntota oblicua:
		$$\lim_{x \to -\infty}((2-x)e^x)=0^{+}$$
		La curva está encima de la asíntota.
		\item Oblicuas: no tiene.
	\end{itemize}
	\item \textbf{6. Corte con los ejes: }
	\begin{itemize}
		\item \textbf{Eje X: }$(2-x)e^{x} = 0 \rightarrow x=2$, raíz simple. Se obtiene el punto $A(2,0)$.
		\item \textbf{Eje Y: }es el punto $B(0,2)$.
		\item \textbf{Signo: }Si $x=0 \rightarrow f(0)=2>0$ (+)
	\end{itemize}
	\item \textbf{7. Máximos y mínimos relativos: }\\
	\begin{itemize}
		\item $C(1,e)$, \textbf{máximo relativo}.\\
	\end{itemize}
	\textbf{Monotonía: }\\
	\begin{itemize}
		\item Creciente: $(-\infty, 1)$
		\item Decreciente: $(1, +\infty)$
	\end{itemize}
	\item \textbf{8. Puntos de inflexión:}\\
	\begin{itemize}
		\item $B(0,2)$, \textbf{punto de inflexión}.\\
	\end{itemize}
	\textbf{Curvatura:}\\
	\begin{itemize}
		\item Convexa $(\bigcup)$: $(-\infty, 0)$.
		\item Cóncava $(\bigcap)$: $(0, +\infty)$
	\end{itemize}
\end{itemize}
\geogebra{cthuzvhc}
\subsubsection{Ejercicios.}
\begin{ex}[sol later]
Comprueba como varía el aspecto de la gráfica dependiendo como se modifica el valor de la base de la función:
\geogebra{qaejsx23}
	\begin{sol}
		Emplea el deslizador que aparece en el anterior applet de Geogebra. 
	\end{sol}
\end{ex}

\begin{ex}[sol later]
	Comprueba como varía el aspecto de la gráfica dependiendo como se modifica el valor de la base de la función:
	\geogebra{amummahk}
	\begin{sol}
		Emplea el deslizador que aparece en el anterior applet de Geogebra. 
	\end{sol}
\end{ex}