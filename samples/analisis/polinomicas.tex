\youtube{YXUf8a3M9vk}
\subsubsection{Modelo de función polinómica}
Analiza y representa la función $y = 2x^2 - \dfrac{x^4}{4}$
\begin{itemize}
	\item \textbf{1. Tipo de función: }polinómica.
	\item \textbf{2. Dominio: }por ser una función polinómica, es toda la recta real $\mathbb{R}$ \\
	Dom(f) = $\mathbb{R} = (-\infty, +\infty)$.
	\item \textbf{3. Continuidad: }por ser polinómica, es toda la recta real $\mathbb{R}$.
	\item \textbf{4. Periodicidad: }no es periódica porque las funciones polinómicas nunca lo son.
	\item \textbf{5. Simetrías: }$f(-x)=2(-x)^2-\dfrac{(-x)^4}{4}=2x^2-\dfrac{x^4}{4}$\\ Se observa que $f(-x)=f(x) \rightarrow$ función par $\rightarrow$ simetría respecto del eje Y.
	\item \textbf{6. Asíntotas: }las funciones polinómicas no tienen asíntotas.
	\item \textbf{7. Corte con los ejes.}
	\begin{itemize}
	\item \textbf{Eje X: }$2x^2-\dfrac{x^4}{4}=0 \rightarrow x=0$, raíz doble; $x_1=2\sqrt{2}$, $x_2=-2\sqrt{2}$, raíces simples.\\
	Se obtienen los puntos $O(0,0)$, $A(-2\sqrt{2},0)$, $B(2\sqrt{2},0)$
	\item \textbf{Eje Y: }es el punto $O(0,0)$
	\item \textbf{Signo: }Si $x=1 \rightarrow f(1)=2-\dfrac{1}{4}=\dfrac{7}{4}>0$ (+)
	\end{itemize}
	\item \textbf{8. Máximos y mínimos:}\\
		$f'(x)=4x-x^3 \rightarrow 4x-x^3=0 \rightarrow x_1 = 0, x_2 = -2, x_3=2$, raíces simples.\\
		$f(x)=2x^2-\dfrac{x^4}{4}$\\
		\begin{itemize}
			\item $f(-2)=4 \rightarrow C(-2,4$)\\
			\item $f(0)=0 \rightarrow O(0,0)$\\
			\item $f(2)=4 \rightarrow D(2,4)$\\
		\end{itemize}
		$f''(x)=4-3x^2$
		\begin{itemize}
			\item $f''(-2)=-8 <0 \rightarrow C(-2,4)$ \textbf{máximo relativo}.\\
			\item $f''(0)=4 >0 \rightarrow O(0,0)$ \textbf{mínimo relativo}.\\
			\item $f''(2)=-8 <0 \rightarrow D(2,4)$ \textbf{máximo relativo}.\\
		\end{itemize}
		\textbf{Monotonía:}\\
		$f'(x)=4x-x^3 \rightarrow$ Si $x=1 \rightarrow f'(x)=4-1=3 > 0$ (+)
	\item \textbf{9. Puntos de inflexión:}\\
	$f''(x)= 4-3x^2 \rightarrow 4-3x^2=0 \rightarrow x_1= -\dfrac{2\sqrt{3}}{3}$, $x_2=\dfrac{2\sqrt{3}}{3}$ raíces simples.\\
	$f(x) = 2x^2-\dfrac{x^4}{4}$
	\begin{itemize}
		\item $f(-\dfrac{2\sqrt{3}}{3}) = \dfrac{20}{9} \rightarrow E(-\dfrac{2\sqrt{3}}{3}, \dfrac{20}{9})$
		\item $f(\dfrac{2\sqrt{3}}{3}) = \dfrac{20}{9} \rightarrow F(\dfrac{2\sqrt{3}}{3}, \dfrac{20}{9})$
	\end{itemize}
	$f'''(x)=-6x$
	\begin{itemize}
		\item $f'''(-\dfrac{2\sqrt{3}}{3}) = 4\sqrt{3} \neq 0 \rightarrow E(-\dfrac{2\sqrt{3}}{3}, \dfrac{20}{9})$, \textbf{punto de inflexión}.
		\item $f'''(\dfrac{2\sqrt{3}}{3}) = -4\sqrt{3} \neq 0 \rightarrow F(\dfrac{2\sqrt{3}}{3}, \dfrac{20}{9})$, \textbf{punto de inflexión}.
	\end{itemize}
	\item \textbf{10. Curvatura:}\\
	$f''(x)=4-3x^2 \rightarrow$ Si $x=0 \rightarrow f''(0)=4 > 0$ (+)
\end{itemize}
\geogebra{ceu6sks6}
