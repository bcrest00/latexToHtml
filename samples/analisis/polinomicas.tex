\subsubsection{Modelo de función polinómica}
Analiza y representa la función $y = 2x^2 - \dfrac{x^4}{4}$
\begin{itemize}
	\item \textbf{1. Tipo de función: }polinómica.
	\item \textbf{2. Dominio: }por ser una función polinómica, es toda la recta real $\mathbb{R}$ \\
	Dom(f) = $\mathbb{R} = (-\infty, +\infty)$.
	\item \textbf{3. Periodicidad: }no es periódica porque las funciones polinómicas nunca lo son.
	\item \textbf{4. Simetrías: }$f(-x)=2(-x)^2-\dfrac{(-x)^4}{4}=2x^2-\dfrac{x^4}{4}$\\ Se observa que $f(-x)=f(x) \rightarrow$ función par $\rightarrow$ simetría respecto del eje Y.
	\item \textbf{5. Asíntotas: }las funciones polinómicas no tienen asíntotas.
	\item \textbf{6. Corte con los ejes.}
	\begin{itemize}
	\item \textbf{Eje X: }$2x^2-\dfrac{x^4}{4}=0 \rightarrow x=0$, raíz doble; $x_1=2\sqrt{2}$, $x_2=-2\sqrt{2}$, raíces simples.\\
	Se obtienen los puntos $O(0,0)$, $A(-2\sqrt{2},0)$, $B(2\sqrt{2},0)$
	\item \textbf{Eje Y: }es el punto $O(0,0)$
	\item \textbf{Signo: }Si $x=1 \rightarrow f(1)=2-\dfrac{1}{4}=\dfrac{7}{4}>0$ (+)
	\end{itemize}
	\item \textbf{7. Máximos y mínimos:}\\
	\begin{itemize}
		\item $C(-2,4)$ \textbf{máximo relativo}.\\
		\item $O(0,0)$ \textbf{mínimo relativo}.\\
		\item $D(2,4)$ \textbf{máximo relativo}.\\
	\end{itemize}
	\textbf{Monotonía:}\\
	\begin{itemize}
		\item Creciente: $(-\infty, -2) \bigcup (0, 2)$
		\item Decreciente: $(-2, 0) \bigcup (2, +\infty)$
	\end{itemize}
	
	\item \textbf{8. Puntos de inflexión:}\\
	\begin{itemize}
		\item $E(-\dfrac{2\sqrt{3}}{3}, \dfrac{20}{9})$, \textbf{punto de inflexión}.
		\item $F(\dfrac{2\sqrt{3}}{3}, \dfrac{20}{9})$, \textbf{punto de inflexión}.
	\end{itemize}
	\item \textbf{9. Curvatura:}\\
	\begin{itemize}
		\item Convexa $(\bigcup)$: $(-\dfrac{2\sqrt{3}}{3}, \dfrac{2\sqrt{3}}{3})$
		\item Cóncava $(\bigcap)$: $(-\infty, -\dfrac{2\sqrt{3}}{3}) \bigcup ( \dfrac{2\sqrt{3}}{3}, +\infty)$
	\end{itemize}
\end{itemize}
\geogebra{ceu6sks6}
\subsubsection{Funciones cuadráticas}
\begin{definition}
En el caso de las funciones polinómicas de grado 2 $f(x)=ax^2+bx+c$ se llaman \textbf{funciones cuadráticas}. Las gráficas de estas funciones son parábolas de eje paralelo al eje de ordenadas y de vértice en el punto 
$$V=(\dfrac{-b}{2a},-\dfrac{b^2-4ac}{4a})$$
Sus ramas van hacia arriba si $a$ es positivo y van hacia abajo si $a$ es negativo.
\end{definition}
Las funciones cuadráticas tienen las siguientes propiedades:
\begin{itemize}
	\item \textbf{Dominio.} El dominio es el conjunto de los números reales: $Dom(f)=\mathbb{R}$.
	\item \textbf{Monotonía.}Pueden ocurrir dos casos:
	\begin{itemize}
		\item Si $a>0$
		\begin{itemize}
			\item $f$ es estrictamente decreciente en $(-\infty, -\dfrac{b}{2a})$
			\item $f$ es estrictamente creciente en $(-\dfrac{b}{2a}, +\infty)$
		\end{itemize}
		\item Si $a<0$
		\begin{itemize}
			\item $f$ es estrictamente decreciente en $(-\infty, -\dfrac{b}{2a})$
			\item $f$ es estrictamente creciente en $(-\dfrac{b}{2a}, +\infty)$ 
		\end{itemize}
	\end{itemize}
	\item \textbf{Máximos y mínimos}
	\begin{itemize}
		\item Si $a>0$, $f$ tiene un mínimo relativo en el vértice.
		\item Si $a<0$, $f$ tiene un máximo relativo en el vértice.
	\end{itemize}
	\item \textbf{Simetría.} La función cuadrática es simétrica respecto de su eje $x=-\dfrac{b}{2a}$. 
\end{itemize}
\subsubsection{Ejercicios}
\begin{ex}[sol later]
	Calcula los vértices de las siguientes parábolas: 
	\begin{itemize}
		\item $f(x)=2x^2-18$
		\item $f(x)=-x^2+4x-6$
		\item $f(x)=x^2+6x$
	\end{itemize}
	\begin{sol}
		\begin{itemize}		
			\item $V(0, -18)$
			\item $V(2, -2)$
			\item $V(-3, -9)$
		\end{itemize}
	\end{sol}
\end{ex}

\begin{ex}[sol later]
	Estudia las propiedades de la siguiente función: $f(x) = -x^2+1$
	\begin{sol}
	Representa la función y corrobora con la gráfica que se cumplen las propiedades que has determinado:\\
	\geogebra{xgyrmrax}
	\end{sol}
\end{ex}
