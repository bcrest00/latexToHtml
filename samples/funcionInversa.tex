
\begin{definition}
Dada una función $f$, se denomina \emph{función inversa} de $f$, si existe, y se simboliza con $f^{-1}$, la función que cumple:
$$f^{-1}(y) = x \longleftrightarrow f(x) = y$$
\end{definition}
La función inversa de $f^{-1}$ es, a su vez, $f: (f^{-1})^{-1}=f$.Por eso decimos, simplemente, que las funciones f y f^{-1} son inversas.
La composición de una función $f$ con su inversa $f^{-1}$ es la \emph{función identidad}, $I(x)=x$:
$$(f^{-1} \circ f)(x) = f^{-1}(f(x)) = I(x)$$
$$(f \circ f^{-1})(x) =f(f^{-1}(x))=I(x)$$

\subsubsection{Obtención de la función inversa}
Para calcular la función inversa de una función dada $f$ debemos de seguir el siguiente procedimiento:
\begin{enumerate}
	\item Se escribe la función con $x$ e $y$.
	\item Se despeja la variable $x$ en función de la variable $y$.
	\item Se intercambian las variables.
\end{enumerate}
\youtube{l6pZGhy0hHc}

\subsubsection{Ejercicios}
\begin{ex}[sol after]
	Halla la función inversa de $f(x)=2x+3$.
	\begin{sol}
		\begin{enumerate}
			\item Intercambiamos las variables $x$ e $y$ en la expresión $y=2x+3$, con lo que resulta $x=2y+3$.
			\item Despejamos $y$, con lo que obtenemos $y=\dfrac{x-3}{2}$. Luego $f^{-1}(x)=\dfrac{x-3}{2}$.
			\item Comprobamos que la composición de las dos funciones hace corresponder a cada $x$ el mismo $x$:
			$$(f^{-1} \circ f)(x) = f^{-1}(f(x)) = f^{-1}(2x+3) = \dfrac{2x + 3 - 3}{2} = x$$
			$$(f \circ f^{-1})(x) = f(f^{-1}(x)) = f(\dfrac{x-3}{2}) =2 \cdot \dfrac{x-3}{2} + 3 = x$$
		\end{enumerate}
	\end{sol}
\end{ex}

\vspace{1cm}

\begin{ex}[sol later]
	En cada caso, calcula la función inversa de la dada:\\
	\begin{itemize}
		\item $f(x) = x^2 -\dfrac{1}{2}$
		\item $h(x) = \log (x)$
		\item $j(x) = \sqrt{x^2+5}$
	\end{itemize}
	\begin{sol}
		\begin{itemize}
			\item $f^{-1}(x) = \sqrt{\dfrac{2x+1}{2}}$
			\item $h^{-1} = e^{x}$
			\item $j^{-1} = \sqrt{x^2-5}$
		\end{itemize}
	\end{sol}
\end{ex}