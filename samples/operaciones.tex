
Sean dos funciones $f$ y $g$ definidas en el mismo subconjunto $D$ de los números reales, se definen las siguientes operaciones entre ellas:
\begin{itemize}
	\item \textbf{Adición}: la suma de $f$ y $g$ es la función $f+g$ que, para cualquier $x \in D$, verifica:
	$$(f+g)(x) = f(x)+g(x)$$
	\item \textbf{Sustracción}: la diferencia de $f$ y $g$ es la función $f-g$ que, para cualquier $x \in D$, verifica:
	$$(f-g)(x) = f(x)-g(x)$$
	\item \textbf{Multiplicación}: el producto de $f$ y $g$ es la función $f \cdot g$ que, para cualquier $x \in D$, verifica:
	$$(f \cdot g)(x) = f(x) \cdot g(x)$$
	\item \textbf{División}: el cociente de $f$ y $g$ es la función $\frac{f}{g}$ que, para cualquier $x \in D$, con $g(x) \neq 0$, verifica:
	$$(\dfrac{f}{g})(x) = \dfrac{f(x)}{g(x)}$$
\end{itemize}
Las operaciones con funciones verifican las mismas propiedades que las operaciones con números reales, ya que se definen utilizando las operaciones con sus imágenes, que son números reales.\\
El dominio de las funciones $f+g$, $f-g$, $f \cdot g$ y $\dfrac{f}{g}$ es la intersección de los dominios de $f$ y $g$, con la salvedad de que en el caso $\dfrac{f}{g}$, los valores de x que anulan el denominador no pertenecen al dominio.\\

EJEMPLO
\youtube{jP1mSfUqpxw}

\subsubsection{Ejercicios}

\begin{ex}[sol later]
	Dadas las funciones $f(x)=5x+6$ y $g(x)=3x^2-4x+8$, calcula la suma $(f+g)(x)$ y la resta $(f-g)(x)$ de las funciones.
	\begin{sol}
		\begin{itemize}
			\item $(f+g)(x) = f(x)+g(x) = (5x+6)+(3x^2-4x+8) = 3x^2+x+14$
			\item $(f-g)(x) = f(x)-g(x) = (5x+6)-(3x^2-4x+8) = -3x^2+9x-2$
		\end{itemize}
	\end{sol}
\end{ex}

\vspace{1cm}

\begin{ex}[sol later]
	Dadas las funciones $f(x)=12x^3+15x^2-6x$ y $g(x)=3x$, calcula el producto $(f \cdot g)(x)$ y la división $(\dfrac{f}{g})(x)$ de las funciones.
	\begin{sol}
		\begin{itemize}
			\item $(f \cdot g)(x) = f(x) \cdot g(x) = (12x^3+15x^2-6x) \cdot(3x) = 36x^4+45x^3-18x^2$
			\item $(\dfrac{f}{g})(x) = \dfrac{f(x)}{g(x)} = \dfrac{(12x^3+15x^2-6x)}{(3x)} = 4x^2+5x-2$
		\end{itemize}
	\end{sol}
\end{ex}

\vspace{1cm}

\begin{ex}[sol later]
	Dadas las funciones $f(x)=\sqrt{x}$, $g(x)=x-4$ y $h(x)=\dfrac{2}{3-x}$, calcula $(f+g)(x), (g+h)(x), (\dfrac{f}{g})(x), (\dfrac{g}{f})(x), (f\cdot g)(x)$ y $(\dfrac{g}{h})(x)$.
	\begin{sol}
		\begin{itemize}
			\item $(f+g)(x) = f(x) + g(x) = \sqrt{x} + (x-4) = \sqrt{x} + x -4$
			\item $(g+h)(x) = g(x) + h(x) = (x-4) + \dfrac{2}{3-x} = x - 4 + \dfrac{2}{3-x}$
			\item $(\dfrac{f}{g})(x) = \dfrac{f(x)}{g(x)} = \dfrac{\sqrt{x}}{x-4}$
			\item $(\dfrac{g}{f})(x) = \dfrac{g(x)}{f(x)} = \dfrac{x-4}{\sqrt{x}}$
			\item $(f\cdot g)(x) = f(x) \cdot g(x) = \sqrt{x} \cdot \dfrac{2}{3-x} = \dfrac{2\sqrt{x}}{3-x}$
			\item $(\dfrac{g}{h})(x) = \dfrac{g(x)}{h(x)} = \dfrac{x-4}{\dfrac{2}{3-x}} = \dfrac{(x-4)(3-x)}{2}$
		\end{itemize}
	\end{sol}
\end{ex}
