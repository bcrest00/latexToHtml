Resuelve los siguientes ejercicios.
\begin{ex}[sol later]
	Dadas las funciones $f(x) = 9x-5$ y $g(x) = 4x+1$, 
	\begin{itemize}
		\item $f(x) + g(x)$
		\item $f(x) - g(x)$
		\item $f(x) \circ g(x)$
		\item $g(x) \circ f(x)$
	\end{itemize}
	\begin{sol}
		\begin{itemize}		
			\item $f(x) + g(x) = 13x-4$
			\item $f(x) - g(x) = 5x-6$
			\item $f(x) \circ g(x) = 9(4x+1)-5$
			\item $g(x) \circ f(x) = 4(9x-5)+1$
		\end{itemize}
	\end{sol}
\end{ex}


\vspace{1cm}


\begin{ex}[sol later]
	Dadas las funciones $f(x) = x^3+2$ y $g(x) =x^2$, 
	\begin{itemize}
		\item $f(x) \cdot g(x)$
		\item $ \dfrac{f(x)}{g(x)}$
		\item $f(x) \circ g(x)$
		\item $g(x) \circ f(x)$
	\end{itemize}

	\begin{sol}
		\begin{itemize}
			\item $f(x) \cdot g(x) = x^5+2x^2$
			\item $ \dfrac{f(x)}{g(x)} = x+\dfrac{2}{x^2}$
			\item $f(x) \circ g(x) = x(x^2)^3+2$
			\item $g(x) \circ f(x) = (x^3+2)^2$
		\end{itemize}
	\end{sol}
\end{ex}

\vspace{1cm}


\begin{ex}[sol later]
	Calcula la función inversa de las siguientes funciones.
	\begin{itemize}
		\item $f(x) = \dfrac{1}{2x-1}$
		\item $g(x) = \dfrac{5x-3}{x}$
		\item $h(x) = \dfrac{1}{3x-2}$
	\end{itemize}

	\begin{sol}
		\begin{itemize}
			\item $f(x)^{-1} = \dfrac{1+x}{2x}$
			\item $g(x)^{-1} = \dfrac{3}{5-x}$
			\item $h(x)^{-1} = \dfrac{2x+1}{3x}$
		\end{itemize}
	\end{sol}
\end{ex}

\vspace{1cm}


\begin{ex}[sol later]
	Dada la función $f(x)  = x(\ln (x))^{2}$, calcula de manera aproximada:
	\geogebra{xa7b2ny7}
	\begin{itemize}
		\item Máximos y mínimos relativos.
		\item Puntos de inflexión.
	\end{itemize}
	\begin{sol}
		\begin{itemize}
			\item Máximo relativo: $(e^{-2}, 4e^{-2})$
			\item Mínimo relativo: $(1,0)$
			\item Punto de inflexión: $(e^{-1}, e^{-1})$
		\end{itemize}
	\end{sol}
\end{ex}

\vspace{1cm}


\begin{ex}[sol later]
	Estudia los intervalos de crecimiento y decrecimiento de la función $f(x) = \dfrac{(x+2)^2}{x+1}$, de manera aproximada:
	\geogebra{y2fpy9dn}
	\begin{sol}
		\begin{itemize}
			\item Crecimiento: $(-\infty, -2) \bigcup (0, +\infty)$
			\item Decrecimiento: $(-2, -1) \bigcup (-1, 0)$
		\end{itemize}
	\end{sol}
\end{ex}

\vspace{1cm}

\begin{ex}[sol later]
	Realice el estudio de las siguientes funciones y esboza sus gráficas:
	\begin{itemize}
		\item $f(x) = \dfrac{x}{x^2+4}$
		\item $e^{-x}(x^2+1)$
	\end{itemize}
	\begin{sol}
		Compruebe las soluciones introduciendo las funciones en el siguiente applet de Geogebra.
		\geogebra[ai=true, stb=true]{tc5pbmae}
	\end{sol}
\end{ex}

\vspace{1cm}

\begin{scq}
	Selecciona cuál de las siguientes funciones es periódica:
	
	\begin{choices}
		\begin{choice}[x]
		$f(x)=4 \cdot \sin (x)$
		\end{choice}
		\begin{choice}
		$f(x) = 3x^2 + \sqrt{3x-4}$
		\end{choice}	
		\begin{choice}
		$f(x) = \dfrac{x+3}{(x-1)^2}$
		\end{choice}
	\end{choices}
	\begin{feedback}
		La segunda opción es una combinación de función polinómica e irracional, y la tercera opción es una función racional.
	\end{feedback}
\end{scq}

\vspace{1cm}

\begin{scq}
	¿Cuál de las siguientes respuestas es correcta para $f(x)$?
	\geogebra{jkv5mdur}
	\begin{choices}
		\begin{choice}
		La función es periódica.
		\end{choice}
		\begin{choice}[x]
		La función presenta un máximo en $(0,0)$.
		\end{choice}	
		\begin{choice}
		La función presenta un punto de inflexión en $(\dfrac{3}{2}, 3)$.
		\end{choice}
	\end{choices}
	\begin{feedback}
		Sólo son periódicas las funciones trigonométricas y el máximo de la función no se encuentra en el punto que se indica en la tercera opción.
	\end{feedback}
\end{scq}

